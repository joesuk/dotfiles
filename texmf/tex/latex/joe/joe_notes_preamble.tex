%\NeedsTeXFormat{LaTeX2e}
%\ProvidesClass{joe_notes}[2014/08/16 Example LaTeX class]


\usepackage{amsmath}
\usepackage{amsfonts}
\usepackage{graphicx}
\usepackage[right=0.5cm,left=0.5cm,top=1.5cm,bottom=2cm]{geometry}
\usepackage[svgnames,dvipsnames]{xcolor}
\usepackage{mathrsfs,amsmath,bm}
\usepackage{amssymb}
\usepackage{titlesec}
\usepackage{tikz-cd}
\usepackage{hyperref}
\usepackage[most]{tcolorbox}
\usepackage{etoolbox}
\usepackage[round]{natbib}
\usepackage{url}
\usepackage{makecell} % allow for line breaks in cell of array
\setlength{\parindent}{0in}
\setlength{\parskip}{0.5em}
%\usepackage{listings}
%\usepackage{color}
%\usepackage{nameref}
\usepackage{ stmaryrd }
%\usepackage{tgbonum}
\usepackage[sfdefault]{noto}
%\usepackage[sfdefault]{roboto}
\usepackage[T1]{fontenc}
\usepackage[shortlabels]{enumitem}


\usepackage[english]{babel}
\usepackage[utf8]{inputenc}
\usepackage{fancyhdr}
%\usepackage[obeyFinal,textsize=scriptsize,shadow]{todonotes}
\usepackage{mathtools}
\usepackage{microtype}
\usepackage[normalem]{ulem}
\usepackage{stmaryrd}
\usepackage{wasysym}
\usepackage{multirow}
\usepackage{prerex}
\usepackage{hyperref}
\usepackage[capitalize,noabbrev]{cleveref}


\usepackage{amsthm}
\theoremstyle{plain}
% \usepackage{titlesec}

% \titleformat{\chapter}[block]
%   {\normalfont\huge\bfseries}{\thechapter}{20pt}{}
% \titlespacing*{\chapter}
%   {0pt}{20pt}{20pt}

\setcounter{tocdepth}{5}
\setcounter{secnumdepth}{5}


%\usepackage{joe_macros}


% section formatting
\titleformat{\section}[block]{\color{violet}\Large\bfseries}{\thesection}{1em}{}
\titleformat{\subsection}[hang]{\color{violet}\large\bfseries}{\thesubsection}{1em}{}

\titleformat{\subsubsection}[hang]{\color{violet}\large\bfseries}{\thesubsubsection}{1em}{}

%% theorems

% solution environment
\renewcommand{\qedsymbol}{$\blacksquare$} % change qed symbol to blacksquare
\newenvironment{sol}
  {\renewcommand\qedsymbol{$\blacksquare$}\begin{proof}[Solution]}
  {\end{proof}}
\newenvironment{solution}
  {\renewcommand\qedsymbol{$\blacksquare$}\begin{proof}[Solution]}
  {\end{proof}}

% solution block
\usepackage{cleveref}
%\usepackage[most]{tcolorbox}
\newtcbtheorem{solb}{Solution}{
                lower separated=false,
                colback=white!80!gray,
colframe=white, fonttitle=\bfseries,
colbacktitle=white!50!gray,
coltitle=black,
enhanced,
breakable,
boxed title style={colframe=black},
attach boxed title to top left={xshift=0.5cm,yshift=-2mm},
}{solb}


% napkin theorem style
\usepackage{thmtools}
\usepackage[framemethod=TikZ]{mdframed}

\newenvironment{boxtype1}
  {%
  \begin{mdframed}[
    linewidth=.5,
    everyline=true,
    %margin=8.5,
    backgroundcolor=gray!2,
    linecolor=red,
    fontcolor=black,
    roundcorner=10pt,
    middlelinewidth=2pt,
    splittopskip=25pt,
    secondextra={
      \node[
        overlay,
        fill=white,
        anchor=west,
        font=\sffamily\large,
        inner xsep=10pt
      ] at ([xshift=10pt]O|-P) {Excursus (Cont.)};
      },
    middleextra={
      \node[
        overlay,
        fill=white,
        anchor=west,
        font=\sffamily\large,
        inner xsep=10pt
      ] at ([xshift=10pt]O|-P) {Excursus (Cont.)};
      }
  ]%
  \fontsize{12}{14}\sffamily\selectfont%
  }
  {\end{mdframed}}

\theoremstyle{definition}
\mdfdefinestyle{mdbluebox}{%
	roundcorner = 5pt,
	frametitlerule=true,
	linewidth=1pt,
	skipabove=12pt,
	innerbottommargin=9pt,
	skipbelow=2pt,
	linecolor=RedViolet,
	backgroundcolor=WildStrawberry!20,
}
\declaretheoremstyle[
	headfont=\sffamily\bfseries\color{RoyalPurple},
	mdframed={style=mdbluebox},
	spaceabove=7pt,
	headpunct={\\[5pt]},
	postheadspace={0pt}
]{thmbluebox}

\mdfdefinestyle{pinegreenbox}{%
	linewidth=0.5pt,
	skipabove=12pt,
	frametitleaboveskip=5pt,
	frametitlebelowskip=0pt,
	skipbelow=2pt,
	frametitlefont=\bfseries,
	innertopmargin=4pt,
	innerbottommargin=8pt,
	linecolor=PineGreen,
	backgroundcolor=SeaGreen!10,
}
\declaretheoremstyle[
	headfont=\bfseries\color{PineGreen},
	mdframed={style=pinegreenbox},
	spaceabove=7pt,
	headpunct={\\[3pt]},
	postheadspace={0pt},
]{thmpinegreenbox}
\declaretheoremstyle[
	headfont=\bfseries\color{RedOrange},
	mdframed={style=mdanswerbox},
	spaceabove=7pt,
	headpunct={\\[3pt]},
	postheadspace={0pt},
]{thmanswerbox}
\declaretheoremstyle[
	headfont=\bfseries\color{RoyalPurple},
	mdframed={style=mdassumptionbox},
	spaceabove=7pt,
	headpunct={\\[3pt]},
	postheadspace={0pt},
]{assumptionbox}



\mdfdefinestyle{mdgreenbox}{%
	skipabove=8pt,
	linewidth=2pt,
	rightline=false,
	leftline=true,
	topline=false,
	bottomline=false,
	linecolor=ForestGreen,
	backgroundcolor=ForestGreen!5,
}
\declaretheoremstyle[
	headfont=\bfseries\sffamily\color{ForestGreen!70!black},
	bodyfont=\normalfont,
	spaceabove=2pt,
	spacebelow=1pt,
	mdframed={style=mdgreenbox},
	headpunct={ --- },
]{thmgreenbox}

\declaretheoremstyle[
	headfont=\bfseries\sffamily\color{ForestGreen!70!black},
	bodyfont=\normalfont,
	spaceabove=2pt,
	spacebelow=1pt,
	mdframed={style=mdgreenbox},
	headpunct={},
]{thmgreenbox*}

\mdfdefinestyle{mdbluebox}{%
	skipabove=8pt,
	linewidth=2pt,
	rightline=false,
	leftline=true,
	topline=false,
	bottomline=false,
	linecolor=Aquamarine,
	backgroundcolor=Aquamarine!5,
}
\declaretheoremstyle[
	headfont=\bfseries\sffamily\color{Aquamarine!70!black},
	bodyfont=\normalfont,
	spaceabove=2pt,
	spacebelow=1pt,
	mdframed={style=mdbluebox},
	headpunct={ --- },
]{thmbluebox}


\mdfdefinestyle{mdblackbox}{%
	skipabove=8pt,
	linewidth=3pt,
	rightline=false,
	leftline=true,
	topline=false,
	bottomline=false,
	linecolor=black,
	backgroundcolor=RedViolet!5!gray!5,
}
\mdfdefinestyle{mdquestionbox}{%
	rightline=false,
	leftline=true,
	topline=false,
	bottomline=false,
	linewidth=5pt,
	skipabove=12pt,
	frametitleaboveskip=5pt,
	frametitlebelowskip=0pt,
	skipbelow=2pt,
	frametitlefont=\bfseries,
	innertopmargin=8pt,
	innerbottommargin=8pt,
	nobreak=true,
	linecolor=WildStrawberry,
	backgroundcolor=WildStrawberry!10,
}
\declaretheoremstyle[
	headfont=\bfseries\large,
	bodyfont=\normalfont,
	spaceabove=2pt,
	spacebelow=1pt,
	mdframed={style=mdquestionbox}
]{questionbox}

\mdfdefinestyle{mdanswerbox}{%
	linewidth=0.5pt,
	skipabove=12pt,
	frametitleaboveskip=5pt,
	frametitlebelowskip=0pt,
	skipbelow=2pt,
	frametitlefont=\bfseries,
	innertopmargin=4pt,
	innerbottommargin=8pt,
	nobreak=true,
	linecolor=Melon,
	backgroundcolor=Dandelion!10,
}
\mdfdefinestyle{mdassumptionbox}{%
	linewidth=0.5pt,
	skipabove=12pt,
	frametitleaboveskip=5pt,
	frametitlebelowskip=0pt,
	skipbelow=2pt,
	frametitlefont=\bfseries,
	innertopmargin=4pt,
	innerbottommargin=8pt,
	nobreak=true,
	linecolor=JungleGreen,
	backgroundcolor=SpringGreen!20,
}

\declaretheoremstyle[
	headfont=\bfseries,
	bodyfont=\normalfont\small,
	spaceabove=0pt,
	spacebelow=0pt,
	mdframed={style=mdblackbox}
]{thmblackbox}


\declaretheorem[style=thmbluebox,name=Theorem]{thm}
\declaretheorem[style=thmbluebox,name=Theorem]{theorem}
\declaretheorem[style=thmbluebox,name=Lemma,sibling=thm]{lemma}
\declaretheorem[style=thmbluebox,name=Lemma,sibling=thm]{lem}
\declaretheorem[style=thmbluebox,name=Proposition,sibling=thm]{prop}
\declaretheorem[style=thmbluebox,name=Corollary,sibling=thm]{cor}
\declaretheorem[style=thmbluebox,name=Theorem,numbered=no]{thm*}
\declaretheorem[style=thmbluebox,name=Lemma,numbered=no]{lemma*}
\declaretheorem[style=thmbluebox,name=Proposition,numbered=no]{prop*}
\declaretheorem[style=thmbluebox,name=Corollary,numbered=no]{cor*}
\declaretheorem[style=thmgreenbox,name=Algorithm]{algo}
\declaretheorem[style=thmgreenbox,name=Algorithm,numbered=no]{algo*}
\declaretheorem[style=thmgreenbox,name=Claim,sibling=thm]{claim}
\declaretheorem[style=thmgreenbox,name=Claim,numbered=no]{claim*}
\declaretheorem[style=thmbluebox,name=Fact,sibling=thm]{fact}
\declaretheorem[style=thmbluebox,name=Fact,numbered=no]{fact*}
\declaretheorem[style=thmpinegreenbox,name=Example]{ex}
\declaretheorem[style=thmpinegreenbox,name=Example,numbered=no]{ex*}
\declaretheorem[style=thmpinegreenbox,name=Example]{example}
\declaretheorem[style=thmpinegreenbox,name=Example,numbered=no]{example*}
\declaretheorem[style=thmanswerbox,name=Answer,numbered=no]{answer}
\declaretheorem[style=thmblackbox,name=Remark]{rmk}
\declaretheorem[style=thmblackbox,name=Remark,numbered=no]{rmk*}
\declaretheorem[style=thmblackbox,name=Remark]{remark}
\declaretheorem[style=thmblackbox,name=Remark,numbered=no]{remark*}
\declaretheorem[style=thmbluebox,name=Proposition,sibling=thm]{proposition}
\declaretheorem[style=thmbluebox,name=Corollary,sibling=thm]{corollary}
\declaretheorem[style=thmpinegreenbox,name=Example]{exercise}
\declaretheorem[style=assumptionbox,name=Assumption]{assumption}
\declaretheorem[style=assumptionbox,name=Assumption,numbered=no]{assumption*}
\declaretheorem[style=questionbox,name=Q.]{ques}


\theoremstyle{definition}
\newtheorem{conj}[thm]{Conjecture}
\newtheorem{definition}[thm]{Definition}
%\newtheorem{assumption}[thm]{Assumption}
%\newtheorem*{assumption*}{Assumption}
\newtheorem{note}[thm]{Notation}
\newtheorem{defn}[thm]{Definition}
%\newtheorem{fact}[thm]{Fact}
%\newtheorem{prop}[prop]{Proposition}
%\newtheorem{answer}[thm]{Answer}
%\newtheorem{ques}[thm]{Question}
\newtheorem{exer}[thm]{Exercise}
\newtheorem{prob}[thm]{Problem}
\newtheorem*{conj*}{Conjecture}
\newtheorem*{defn*}{Definition}
\newtheorem*{note*}{Notation}
\newtheorem*{fact*}{Fact}
%\newtheorem*{answer*}{Answer}
\newtheorem*{ques*}{Question}
\newtheorem*{exer*}{Exercise}
\newtheorem*{prob*}{Problem}

% cleveref setup
\Crefname{defn}{Defn.}{Defn.}
\Crefname{definition}{Defn.}{Defn.}
\Crefname{rmk}{Rmk.}{Rmk.}
\Crefname{prop}{Prop.}{Prop.}
\Crefname{thm}{Thm.}{Thm.}
\Crefname{cor}{Cor.}{Cor.}
\Crefname{lem}{Lem.}{Lem.}
\Crefname{lemma}{Lem.}{Lem.}
\Crefname{algo}{Alg.}{Alg.}
\Crefname{ex}{Ex.}{Ex.}
\Crefname{answer}{Answer}{Answers}
\Crefname{ques}{Ques.}{Ques.}
\Crefname{prob}{Prob.}{Prob.}
\Crefname{assumption}{Assumption}{Assumptions}
\Crefname{note}{Notation}{Notations}
\Crefname{fact}{Fact}{Facts}
\Crefname{exer}{Exer.}{Exer.}
\Crefname{conj}{Conj.}{Conj.}
\Crefname{claim}{Claim}{Claims}
\Crefname{subsection}{Subsec.}{Subsec.}
\Crefname{section}{Sec.}{Sec.}
\Crefname{table}{Table}{Tables}
\crefformat{equation}{(#2#1#3)}

% algorithm2e
\usepackage[algo2e, vlined, ruled, linesnumbered]{algorithm2e}
% algorithm2e commenting
%\newcommand{\com}[1]{{\color{red}\textbf{[[} #1 \textbf{]]}}}
%\newcommand\mycommfont[1]{\footnotesize\textcolor{black}{#1}}
\newcommand\mycommfont[1]{\textcolor{blue}{#1}}
\SetCommentSty{mycommfont}

\usepackage{etoolbox}  % patch def of algorithmic environment
\makeatletter
\patchcmd{\algorithmic}{\addtolength{\ALC@tlm}{\leftmargin} }{\addtolength{\ALC@tlm}{\leftmargin}}{}{}
\makeatother



\newenvironment{protocol}[1][htb]
  {\renewcommand{\algorithmcfname}{Protocol}% Update algorithm name
   \begin{algorithm2e}[#1]%
   }{\end{algorithm2e}}


\newenvironment{exampl}[1][htb]
  {%\refstepcounter{examcounter}\par\medskip
  \renewcommand{\algorithmcfname}{Algorithm}
   \begin{algorithm2e}[#1]%
   }{\end{algorithm2e}}


\newenvironment{procedures}[1][htb]
  {\renewcommand{\algorithmcfname}{Procedure}% Update algorithm name
   \begin{algorithm2e}[#1]%
   }{\end{algorithm2e}}

\def\QED{{\phantom{x}} \hfill \ensuremath{\rule{1.3ex}{1.3ex}}}
\newcommand{\extraproof}[1]{\rm \trivlist

\item[\hskip \labelsep{\bf Proof of #1. }]}
\def\endextraproof{\QED \endtrivlist}

\newenvironment{rtheorem}[3][]{

\bigskip

\noindent \ifthenelse{\equal{#1}{}}{\bf #2 #3}{\bf #2 #3 (#1)}
\begin{it}
}{\end{it}}

\newcommand{\nonl}{\renewcommand{\nl}{\let\nl}}
\def\NoNumber#1{{\def\alglinenumber##1{}\State #1}\addtocounter{ALG@line}{-1}}
\newcommand{\obs}{o}

% enable do while
\SetKwRepeat{Do}{do}{while}


% if you use cleveref..
%\usepackage{nameref}
%\usepackage[capitalize,noabbrev]{cleveref}
\crefname{algocf}{Alg.}{Alg.}
\Crefname{algocfproc}{Alg.}{Alg.}

% fix algorithm2e labelling by cleverref
\crefalias{AlgoLine}{line}%
\makeatletter
\let\cref@old@stepcounter\stepcounter
\def\stepcounter#1{%
  \cref@old@stepcounter{#1}%
  \cref@constructprefix{#1}{\cref@result}%
  \@ifundefined{cref@#1@alias}%
    {\def\@tempa{#1}}%
    {\def\@tempa{\csname cref@#1@alias\endcsname}}%
  \protected@edef\cref@currentlabel{%
    [\@tempa][\arabic{#1}][\cref@result]%
    \csname p@#1\endcsname\csname the#1\endcsname}}
\makeatother

% colors and hypersetup
\definecolor{aqua}{rgb}{0.0, 1.0, 1.0}
\definecolor{caribbeangreen}{rgb}{0.0, 0.8, 0.6}
\definecolor{azure}{rgb}{0.0, 0.5, 1.0}
\definecolor{charcoal}{rgb}{0.21, 0.27, 0.31}
\hypersetup{
	%backref=true,
    %pagebackref=true,
    %hyperindex=true,
    colorlinks=true,
    breaklinks=true,
	citecolor=Mulberry,
	filecolor=black,
	linkcolor=azure,
	urlcolor=PineGreen,
	linkbordercolor=blue
}


% table of contents spacing
\usepackage{tocloft}
\setlength{\cftbeforesecskip}{5pt}

% redefine title command
\makeatletter
\def\@maketitle{%
  \begin{center}%
  { \begin{tcolorbox}[colframe=caribbeangreen,colback=caribbeangreen!10]
		  \LARGE\bf\centering\@title \color{charcoal}
\end{tcolorbox}
}%
  \end{center}%
  }
\makeatother

\endinput
